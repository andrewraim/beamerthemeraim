% First line is a Workaround to make example compile on math.umbc.edu machines
% Shouldn't be needed in general
% See: http://old.nabble.com/incompatible-list-can%27t-be-unboxed-with-every-beamer-document-td24108128.html
\RequirePackage{atbegshi}

\documentclass[8pt,final,hyperref={pdfpagelabels=false},xcolor=dvipsnames]{beamer}
\mode <presentation>

\usetheme{foiltex}

\usepackage{epsfig}
\usepackage{geometry}
\usepackage{color}
\usepackage{graphicx}
\usepackage{hyperref}
\usepackage{verbatim}


\begin{document}

\title[Example of FoilTex-like Theme]{Sample presentation using a FoilTeX-like Theme}
\author[M.Klein, A. Raim]{Martin D. Klein, Andrew M. Raim}
\date{2010}
\institute[UMBC]{
Department of Mathematics and Statistics\\
University of Maryland, Baltimore County\\
\url{mklein1@umbc.edu, araim1@umbc.edu}}

%%%%%%%%%%%%%%%%%%%%%%%%%%%%%%%%%%%%%%%%%%%%%%
\begin{frame}[plain]
\maketitle
\end{frame}


%%%%%%%%%%%%%%%%%%%%%%%%%%%%%%%%%%%%%%%%%%%%%%
\section{Note}
\begin{frame}{Note}
The rest of this example presentation mimics the slides from
\url{http://math.arizona.edu/~swig/documentation/powerwhat/}
(see foiltex-example.pdf). The idea was to create a similar look.
\end{frame}


%%%%%%%%%%%%%%%%%%%%%%%%%%%%%%%%%%%%%%%%%%%%%%
\section{Motivation}
\begin{frame}{Motivation}
\begin{Witemize}

\item (m a t h) Graduate students and professors use Latex

\item Create a .pdf presentation, compatible with different operating systems

\end{Witemize}
\end{frame}


%%%%%%%%%%%%%%%%%%%%%%%%%%%%%%%%%%%%%%%%%%%%%%
\subsection{Mechanics}
\begin{frame}{Mechanics}
\begin{Witemize}[10pt]
\item write/rewrite file.tex\\
\item latex file.tex (creates file.dvi, if everything works)\\
\item dvips file.dvi (-o) (creates file.ps)\\
\item ps2pdf file.ps (creates file.pdf !)\\
\end{Witemize}

Suggestions for editing: use \textbf{Kile}, WinEdt, etc.
\end{frame}

%%%%%%%%%%%%%%%%%%%%%%%%%%%%%%%%%%%%%%%%%%%%%%
\section{Latex presentations}
\begin{frame}{Overview}
\begin{Witemize}
\item Main classes for Latex presentations:\\
\hspace*{.5cm}foiltex, prosper, and beamer\\
\item Setting up the tex files for each\\
\item Features and layouts\\
\item References for further learning\\

\end{Witemize}
\end{frame}

%%%%%%%%%%%%%%%%%%%%%%%%%%%%%%%%%%%%%%%%%%%%%%%
\begin{frame}[fragile]{Foiltex Setup}

\begin{verbatim}
\documentclass[20pt,landscape,footrule]{foils}
\begin{document}
\title{ Title of Presentation }
\author{ Author's name }
\date{ date of Presentation }
\maketitle
\MyLogo{ text for footer or header }
\foilhead{ title of slide }
contents of slide
\foilhead{ title of slide }
contents of slide
...
\end{document}
\end{verbatim}
\end{frame}

%%%%%%%%%%%%%%%%%%%%%%%%%%%%%%%%%%%%%%%%%%%%%%
\begin{frame}[fragile]{Slide Example}
\begin{verbatim}
\foilhead{Definition from (college) Algebra}
\begin{displaymath}
Crazy math goes here!
\end{displaymath}
\end{verbatim}
\end{frame}

%%%%%%%%%%%%%%%%%%%%%%%%%%%%%%%%%%%%%%%%%%%%%%%
\begin{frame}{Definition from (college) Algebra}
\textbf{Definition}: The $p^{\textrm{th}}$ supported deRham cohomology group of $M$
\begin{displaymath}
H^p_c(M) = \frac{\textrm{Ker}[d: \mathcal{A}^p_c(M) \longrightarrow \mathcal{A}^{p+1}_c(M)]}{\textrm{Im}[d: \mathcal{A}^{p-1}_c(M) \longrightarrow \mathcal{A}^{p}_c(M)]}
\end{displaymath}
\end{frame}

%%%%%%%%%%%%%%%%%%%%%%%%%%%%%%%%%%%%%%%%%%%%%%%
\begin{frame}{Advantages and Disadvantages}
\begin{Witemize}
\item Easiest, Fastest to use\\
\item Simpleness may be limiting\\
\item Boring to look at ?\\
\end{Witemize}
\end{frame}

%%%%%%%%%%%%%%%%%%%%%%%%%%%%%%%%%%%%%%%%%%%%%%%%%
\begin{frame}{For more information}
\begin{Witemize}
\item Documentation for Foiltex\\
\texttt{http://www.tex.ac.uk/tex-archive/nonfree/macros/latex/contrib/}\\
\texttt{foiltex/foiltex.pdf}\\[.5cm]
\item Get what you can from an internet search
\end{Witemize}
\end{frame}
\end{document}

